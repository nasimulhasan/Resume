%%%%%%%%%%%%%%%%%%%%%%%%%%%%%%%%%%%%%%%%%
% Medium Length Professional CV
% LaTeX Template
% Version 2.0 (8/5/13)
%
% This template has been downloaded from:
% http://www.LaTeXTemplates.com
%
% Original author:
% Trey Hunner (http://www.treyhunner.com/)
%
% Important note:
% This template requires the resume.cls file to be in the same directory as the
% .tex file. The resume.cls file provides the resume style used for structuring the
% document.
%
%%%%%%%%%%%%%%%%%%%%%%%%%%%%%%%%%%%%%%%%%

%----------------------------------------------------------------------------------------
%	PACKAGES AND OTHER DOCUMENT CONFIGURATIONS
%----------------------------------------------------------------------------------------

\documentclass{resume} % Use the custom resume.cls style

\usepackage[left=0.75in,top=0.6in,right=0.75in,bottom=0.6in]{geometry} % Document margins
\newcommand{\tab}[1]{\hspace{.2667\textwidth}\rlap{#1}}
\newcommand{\itab}[1]{\hspace{0em}\rlap{#1}}
\name{Siddhant Srivastava} % Your name
% Your address
%\address{123 Pleasant Lane \\ City, State 12345} % Your secondary addess (optional)
\address{(+91)7388617414 \\ siddhant.srivastava11@gmail.com} % Your phone number and email
\address{https://in.linkedin.com/in/siddhantsrivastava1 \\ https://github.com/sidsriv}
%\address{\textbf{Undergraduate Researcher (Data Science)}}
\address{\textbf{Pre-Final Year Student (B.Tech-M.Tech dual degree programme)}}
\begin{document}

%----------------------------------------------------------------------------------------
%	EDUCATION SECTION
%----------------------------------------------------------------------------------------

\begin{rSection}{Education}
\quad \\
{\bf ABV Indian Institute of Information Technology \& Management, Gwalior} %\hfill {\em July 2008 - Present} 
\\ Integrated B.Tech-M.Tech degree in Information Technology  \hfill {\em July 2014 - Present} 
\\Junior Undergraduate  \hfill { Overall GPA: 8.22/10}
%\\Relevant Coursework: bla, blabla, blablabla

{\bf Cathedral Sr. Sec. School, Lucknow} %\hfill {\em July 2008 - Present} 
\\ Intermediate (C.B.S.E.)    \hfill {\em Score: 94.4\%, Year:2014} 
\\High School (C.B.S.E.)  \hfill {Score: 9/10, Year:2012}

%Minor in Linguistics \smallskip \\
%Member of Eta Kappa Nu \\
%Member of Upsilon Pi Epsilon \\

\end{rSection}
%----------------------------------------------------------------------------------------
%	Research interest SECTION
%----------------------------------------------------------------------------------------

\begin{rSection}{Research Interests}

\begin{itemize}
\item \textbf{Data Science}: Machine Learning, Statistics, Artificial Intelligence, Natural Language Processing, Deep Learning, Nature Inspired Algorithms. 
\item \textbf{Bioinformatics}: Structural Biology, Genomic Data Analysis, Mathematical modelling of Biological Systems.
\end{itemize}
\end{rSection}


%----------------------------------------------------------------------------------------
%	WORK EXPERIENCE SECTION
%----------------------------------------------------------------------------------------

\begin{rSection}{Research Experience}

\begin{rSubsection}{Institute for Bioinformatics and Applied Biotechnology}{May'16 - July'16}{\textbf{Construction and analysis of protein-protein interaction network in \textit{Leishmania donovani}
} \\
 \emph{\it Supervisor: Dr. Shubhada Hegde} \\[-0.6cm]}{}
	\item \it {Research objective} : Comparison of cross study gene expression data (Agilent Microarray) for drug sensitive and drug resistant \textit{L.donovani}.
    \item \it {Individual Responsibility}: Cross-platform normalized gene expressions from different experiments. Calculated Pearson correlation, mutual information score for gene pairs and implemented Principal component analysis, K-means clustering algorithms on normalized datasets.	
	\item \it {Accomplishments:} Performed correlation and mutual information analysis on gene pairs to study pattern similarity in gene expression and further employed cluster algorithms. Normalized 22 gene expression samples from 6 different experiments available in NCBI GEO database.
	\item Tools used : Python Scientific stack (Numpy, Pandas, Scikit-learn, Matplotlib), R and Rstudio (Bioconductor).
\end{rSubsection}

\begin{rSubsection}{ABV Indian Institute of Information Technology \& Management}{May'17 - July'17}{\textbf{Hybrid evolutionary approach
for Devanagari Handwritten numeral recognition using Convolutional Neural Network} \\
 \emph{\it Supervisor: Prof.Anupam Shukla, \hspace{0.2cm} Dr.Ritu Tiwari} \\[-0.6cm]}{}
	\item \it {Research objective} : To develop a novel approach for training Convolutional Neural Network using initial weights based on Genetic Algorithms for fast optimization. 
    \item \it {Individual Responsibility}: Pre-processed 25,000 images of isolated handwritten Hindi characters using binarization, blurring, cropping, size normalization. Implemented vanilla CNN architechture with integrated genetic algorithm optimization using python.
	\item \it {Accomplishments:} Achieved state-of-the-art accuracy of 96.83\% on test data using our model and compared our results with standard classification algorithms like logistic regression, random forest and support vector machine.
	\item Tools used : Python 3.5, Numpy, Matplotlib, Scikit-learn.
\end{rSubsection}



\end{rSection}

%----------------------------------------------------------------------------------------
%	TECHNICAL STRENGTHS SECTION
%----------------------------------------------------------------------------------------
\begin{rSection}{Certifications}
\begin{itemize}
\item \noindent \textbf{Specializations}
\begin{enumerate}
\item \noindent Python for Everybody, a 5-course specialization by University of Michigan on Coursera. Specialization Certificate earned on April 6, 2016.
\item \noindent Genomic Data Science, an 8-course specialization by Johns Hopkins University on Coursera. Specialization Certificate earned on October 4, 2016.
\item \noindent Data Science , a 10-course specialization by Johns Hopkins University on Coursera. Specialization Certificate earned on November 18, 2016.
\item \noindent Machine Learning, a 4-course specialization by University of Washington on Coursera. Specialization Certificate earned on September 16, 2016.
\item \noindent Data Mining, a 5-course specialization by University of Illinois on Coursera. Specialization Certificate earned on June 7, 2017.
\end{enumerate}
\item \noindent \textbf{Relevant Courses}
\begin{enumerate}
\item  Bioinformatic Methods I by University of Toronto on Coursera. Certificate earned on December 18, 2015.
\item  Bioinformatic Methods II by University of Toronto on Coursera. Certificate earned on January 17, 2016.
%\item  Biology Meets Programming: Bioinformatics for Beginners by University of California, San Diego on Coursera. Certificate earned on March 14, 2016.
%\item Automata by University of Stanford on Coursera. Certificate earned on November 12, 2015.
\item Data Analytics and Statistical Inference by Duke University on Coursera. Certificate earned on November 27, 2015.
%\item Practical Predictive Analytics: Models and Methods by University of Washington on Coursera. Certificate earned on August 30, 2016.
%\item Data Manipulation at Scale: Systems and Algorithms by University of Washington on Coursera. Certificate earned on August 11, 2016.
%\item Advanced Linear Models for Data Science 1: Least Squares by Johns Hopkins University on Coursera. Certificate earned on June 21, 2016.
%\item Probability by University of Pennsylvania on Coursera. Certificate earned on November 29, 2015.
%\item Introduction to Big Data (2015) by University of California, San Diego on Coursera. Certificate earned on April 13, 2016.
%\item Hadoop Platform and Application Framework by University of California, San Diego on Coursera. Certificate earned on July 3, 2016.
\end{enumerate}
\end{itemize}
\end{rSection}
\begin{rSection}{Technical Skills}

\begin{tabular}{ @{} >{\bfseries}l @{\hspace{6ex}} l }
Computer Languages &  C/C++, Python, R, MySQL, Julia. \\
Software \& Tools & MATLAB, Shell, \LaTeX, Anaconda, Rstudio, Jupyter notebook.\\
\end{tabular}

\end{rSection}


%	EXAMPLE SECTION
%----------------------------------------------------------------------------------------
% Remove if you think you dont have enough points
\begin{rSection}{Academic Achievements} \itemsep -2pt
\item A.I.S.S.C.E 2014 Topper, Cathedral Sen. Sec. School
\item Secured 99\% marks in Chemistry, A.I.S.S.C.E 2014
\end{rSection}

%--------------uncomment only when you have decent positions of responsibility------------------------
\begin{rSection}{POSITION OF RESPONSIBILITY}

\begin{rSubsection}{Event Organizer}{October 2016 - November 2016}{Otaku - Anime Quiz}{Infotsav 2016}
\item Organized Manga trivia quiz at Infotsav, the annual technical festival at ABV-IIITM Gwalior 

\end{rSubsection}

%------------------------------------------------

%\begin{rSubsection}{Justice league member}{April 2015 - Present}{Internship Coordinator}{ABV-IIITM}
%\item stuck kryptonite uo in superman's ass 

%\end{rSubsection}

%------------------------------------------------



\end{rSection}
%--------------write co-curricular activities------------------------
%----------------------------------------------------------------------------------------
\begin{rSection}{Extra-Curricular} \itemsep -3pt
\begin{enumerate}
%\item From the Big Bang to Dark Energy by The University of Tokyo on Coursera. Certificate earned on November 14, 2015.
%\item The Ancient Greeks by Wesleyan University on Coursera. Certificate earned on January 24, 2016.
%\item The Power of Microeconomics: Economic Principles in the Real World by University of California, Irvine on Coursera. Certificate earned on March 5, 2016
%\item The Power of Macroeconomics: Economic Principles in the Real World by University of California, Irvine on Coursera. Certificate earned on April 12, 2016
%\item Game Theory by Stanford Univeristy on Coursera. Certificate earned on November 16, 2015.
%\item Introduction to Mathematical thinking by University of Stanford on Coursera. Certificate earned on November 16, 2015.
%\item Introduction to Logic by University of Stanford on Coursera. Certificate earned on December 22, 2015.
%\item Internet History, Technology, and Security by University of Michigan on Coursera. Certificate earned on January 24, 2016.
%\item Duolingo: Intermediate German Fluency.
\item Active Member of Coursera's Beta Tester Community, my task is to review upcoming courses before making it available to students world wide.
\item Active Member of Coursera's Data Science Community, a network of data science professionals and enthusiasts where we discuss and keep up to date with latest news, trends and technologies in data science.  
\end{enumerate}
\end{rSection}
\newpage
\begin{rSection}{Recommendations}
\textbf{Prof. Anupam Shukla} \hspace{1.3cm} \textbf{Dr.Ritu Tiwari} \hspace{4cm} \textbf{Dr. Shubhada Hegde}\\
Professor, ABV-IIITM Gwalior \hspace{0.3cm}Associate Professor, ABV-IIITM Gwalior \hspace{0.1cm} Faculty Scientist, IBAB\\
anupamshukla@iiitm.ac.in \hspace{1.1cm} ritutiwari@iiitm.ac.in \hspace{3.5cm} shubhada@ibab.ac.in
\end{rSection}
\end{document}
\grid
\grid
\grid
\grid
